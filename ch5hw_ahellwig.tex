%
% Module 3 Chapter 5 Homework
% CSC160-C00: Computer Science I (C++)
% Author: Ashton Hellwig
%


\documentclass[a4paper]{article}


    % Packages
    \usepackage[english]{babel}       % Internationalization
    \usepackage{soul}                 % Highlighting
    \usepackage{hyperref}             % Links (internal and external)
    \usepackage{fancyhdr}             % Headers and footers
    \usepackage[dvipsnames]{xcolor}   % Text Colors
    \usepackage{listings}             % Code Snippets
    \usepackage{algorithmicx}         % Algorithmic notation support
    \usepackage{algpseudocode}        % Algorithmic notation environments
    \usepackage{enumitem}             % Ordered lists
    \usepackage{geometry}             % Page layout
    \usepackage{graphicx}             % Image support
    \usepackage[toc, page]{appendix}  % Appendix


    % Colors
    \newcommand{\commentstylecolor}{\color{Gray}}
    \newcommand{\keywordstylecolor}{\color{MidnightBlue}}
    \newcommand{\stringstylecolor}{\color{ForestGreen}}
    \newcommand{\questioninput}{\color{Red}}
    \newcommand{\answertcolor}{\color{Green}}
    \newcommand{\myanswer}{\answertcolor{\hl}}


    % Image Directory
    \graphicspath{ {screenshots/} }


    % Hyperlink Setup
    \hypersetup{
      colorlinks = true,
      urlcolor = blue,
      linkcolor = blue
    }


    % Syntax-Highlighting for Code Snippets
    \lstset{
      backgroundcolor=\color{white},
      breaklines=true,%
      captionpos=b,%
      frame=tb,%
      tabsize=4,%
      numbers=left,%
      showstringspaces=false,%
      commentstyle=\commentstylecolor,%
      keywordstyle=\keywordstylecolor,%
      stringstyle=\stringstylecolor%
    }


    % Page Configuration
    %% Style
    \pagestyle{fancy}

    %% Layout
    \geometry{%
    a4paper,%
    top=2.5cm,%
    bottom=2.5cm,%
    left=2.5cm,%
    right=2.5cm%
    }
    \setlength{\headheight}{12pt}
    \setlength{\floatsep}{12pt}

    %% Title page
    \title{Module 3 Chapter 5 Homework}
    \author{Ashton Hellwig}
    \date\today
    \setcounter{tocdepth}{3}

    %% Subsequent pages
    \lhead{CSC160}
    \rhead{Computer Science I (C++)}
    \lfoot{M2C4HW}
    \rfoot{A. Hellwig}


    % Document Content
  \begin{document}
    \maketitle
    \tableofcontents
    \listoffigures
    \newpage


    % Question 1
    \section{QUESTION 1}
      Suppose that the input is \texttt{3 4 6 7 2 -1}. What is the outputof the
        following code?
      \begin{lstlisting}[language=c++]
int num;
int sum;

cin >> num;
sum = num;

while (num != -1)
{
    sum = sum + 2 * num;
    cin >> num;
}

cout << "Sum = " << sum << endl;
      \end{lstlisting}
      % Question 1 Solution
      \subsection{Solution}
        TODO: SOLUTION
        \begin{verbatim}

        \end{verbatim}
        \begin{figure}[h]
          \caption{Question 1 Solution}
          \centering
          %\includegraphics[width=\textwidth]{Q1F1}
          \label{fig:q1}
        \end{figure}


    % Appendices
    \newpage
    \appendix

    \section{General Functions}
      Since I need my source files to work on both my machine running Windows and
        Arch Linux, I need cross-platform functions to use. I can never seem to
        get \texttt{system(``pause'')} to work on any OS other than Windows. To
        ratify this, I created the following function (under its own class
        and with its own header file) to do relativly the same thing. It is below.
        \begin{figure}[h]
          \caption{GenFuncs.cpp}
          \begin{lstlisting}[language=c++]
#include "GenFuncs.h"
#include <iostream>

int GenFuncs::pauseprompt()
{
    std::cout << "Press enter to continue..." << std::endl;
    std::cin.ignore();
    return 0;
}
          \end{lstlisting}
        \end{figure}
        \begin{figure}[h]
          \caption{GenFuncs.h}
          \begin{lstlisting}[language=c++]
#ifndef GEN_FUNCS_H_INCLUDED

class GenFuncs
{
public:
    static int pauseprompt();
};

#endif GEN_FUNCS_H_INCLUDED
          \end{lstlisting}
        \end{figure}
        \begin{figure}[h]
          \caption{GenFuncs::pauseprompt();}
          \centering
          %\includegraphics[width=\textwidth]{}
        \end{figure}


    % Question 3 Appendix
    \newpage
    \section{Question TODO: Appendix}



  \end{document}